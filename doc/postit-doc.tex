% !TeX TXS-program:compile = txs:///arara
% arara: pdflatex: {shell: yes, synctex: no, interaction: batchmode}
% arara: pdflatex: {shell: yes, synctex: no, interaction: batchmode} if found('log', '(undefined references|Please rerun|Rerun to get)')

\documentclass[french,a4paper,11pt]{article}
\usepackage[margin=2cm,includefoot]{geometry}
\def\TPversion{0.1}
\def\TPdate{31 Mai 2023}
\usepackage[utf8]{inputenc}
\usepackage[T1]{fontenc}
\usepackage{amsmath,amssymb}
\usepackage{postit}
\usepackage{awesomebox}
\usepackage{fontawesome5}
\usepackage{footnote}
\makesavenoteenv{tabular}
\usepackage{enumitem}
\usepackage{tabularray}
\usepackage{wrapstuff}
\usepackage{lipsum}
\usepackage{fancyvrb}
\usepackage{fancyhdr}
\fancyhf{}
\renewcommand{\headrulewidth}{0pt}
\lfoot{\sffamily\small [postit]}
\cfoot{\sffamily\small - \thepage{} -}
\rfoot{\hyperlink{matoc}{\small\faArrowAltCircleUp[regular]}}

%\usepackage{hvlogos}
\usepackage{hologo}
\providecommand\tikzlogo{Ti\textit{k}Z}
\providecommand\TeXLive{\TeX{}Live\xspace}
\providecommand\PSTricks{\textsf{PSTricks}\xspace}
\let\pstricks\PSTricks
\let\TikZ\tikzlogo
\newcommand\TableauDocumentation{%
	\begin{tblr}{width=\linewidth,colspec={X[c]X[c]X[c]X[c]X[c]X[c]},cells={font=\sffamily}}
		{\LARGE \LaTeX} & & & & &\\
		& {\LARGE \hologo{pdfLaTeX}} & & & & \\
		& & {\LARGE \hologo{LuaLaTeX}} & & & \\
		& & & {\LARGE \TikZ} & & \\
		& & & & {\LARGE \TeXLive} & \\
		& & & & & {\LARGE \hologo{MiKTeX}} \\
	\end{tblr}
}

\usepackage{hyperref}
\urlstyle{same}
\hypersetup{pdfborder=0 0 0}
\setlength{\parindent}{0pt}
\definecolor{LightGray}{gray}{0.9}

\usepackage{babel}
\AddThinSpaceBeforeFootnotes
\FrenchFootnotes

\tcbuselibrary{listingsutf8}
\newtcblisting{DemoCode}[1][]{%
	enhanced,width=0.95\linewidth,center,%
	bicolor,size=title,%
	colback=cyan!2!white,%
	colbacklower=cyan!1!white,%
	colframe=cyan!75!black,%
	listing options={%
		breaklines=true,%
		breakatwhitespace=true,%
		style=tcblatex,basicstyle=\small\ttfamily,%
		tabsize=4,%
		commentstyle={\itshape\color{gray}},
		keywordstyle={\color{blue}},%
		classoffset=0,%
		keywords={},%
		alsoletter={-},%
		keywordstyle={\color{blue}},%
		classoffset=1,%
		alsoletter={-},%
		morekeywords={center,right,justify,left,\lipsum},%
		keywordstyle={\color{violet}},%
		classoffset=2,%
		alsoletter={-},%
		morekeywords={PostIt,\MiniPostIt},%
		keywordstyle={\color{green!50!black}},%
		classoffset=3,%
		morekeywords={Couleur,CouleurAttache,Attache,Largeur,Hauteur,Inclinaison,Ombre,Coin,DecalAttache,AlignementH,AlignementV,AlignementPostIt,Bordure},%
		keywordstyle={\color{orange}}
	},%
	#1
}

\tcbset{vignettes/.style={%
	nobeforeafter,box align=base,boxsep=0pt,enhanced,sharp corners=all,rounded corners=southeast,%
	boxrule=0.75pt,left=7pt,right=1pt,top=0pt,bottom=0.25pt,%
	}
}

\tcbset{vignetteMaJ/.style={%
	fontupper={\vphantom{pf}\footnotesize\ttfamily},
	vignettes,colframe=purple!50!black,coltitle=white,colback=purple!10,%
	overlay={\begin{tcbclipinterior}%
			\fill[fill=purple!75]($(interior.south west)$) rectangle node[rotate=90]{\tiny \sffamily{\textcolor{black}{\scalebox{0.66}[0.66]{\textbf{MàJ}}}}} ($(interior.north west)+(5pt,0pt)$);%
	\end{tcbclipinterior}}
	}
}

\newcommand\Cle[1]{{\bfseries\sffamily\textlangle #1\textrangle}}
\newcommand\cmaj[1]{\tcbox[vignetteMaJ]{#1}\xspace}

\begin{document}

\setlength{\aweboxleftmargin}{0.07\linewidth}
\setlength{\aweboxcontentwidth}{0.93\linewidth}
\setlength{\aweboxvskip}{8pt}

\pagestyle{fancy}

\thispagestyle{empty}

\vspace{2cm}

\begin{center}
	\begin{minipage}{0.75\linewidth}
	\begin{tcolorbox}[colframe=yellow,colback=yellow!15]
		\begin{center}
			\begin{tabular}{c}
				{\Huge \texttt{postit}}\\
				\\
				{\LARGE Des petits Post-It,} \\
				\\
				{\LARGE avec \textsf{tcolorbox}.} \\
			\end{tabular}
			
			\bigskip
			
			{\small \texttt{Version \TPversion{} -- \TPdate}}
		\end{center}
	\end{tcolorbox}
\end{minipage}
\end{center}

\begin{center}
	\begin{tabular}{c}
	\texttt{Cédric Pierquet}\\
	{\ttfamily c pierquet -- at -- outlook . fr}\\
	\texttt{\url{https://github.com/cpierquet/postit}}
\end{tabular}
\end{center}

\vspace{0.25cm}

{$\blacktriangleright$~~Placer et personnaliser des Post-It ou des \textit{mini-}Post-It.}

\vspace{0.25cm}

{$\blacktriangleright$~~Gestion de la taille, de l'inclinaison, de la décoration.}

\vspace{1cm}

\begin{PostIt}<center>
	Ceci est un petit Post-It ! Pour rappeler par exemple que \[(a+b)^2=a^2+2ab+b^2.\]
\end{PostIt}

\begin{PostIt}[Largeur=8cm,Couleur=orange,Attache=Non,Inclinaison=-5,Coin,AlignementPostIt=center]
\lipsum[1][1-4]
\end{PostIt}
\hfill
\begin{PostIt}[Hauteur=6cm,AlignementV=center,Couleur=pink,Attache=Trombone,CouleurAttache=blue,Inclinaison=15,Coin,AlignementPostIt=center]
\lipsum[1][1-4]
\end{PostIt}

\vspace{0.5cm}

%\hfill{}\textit{Merci à Denis Bitouzé et à Gilles Le Bourhis pour leurs retours et idées !}

\smallskip

\vfill

\hrule

\medskip

\TableauDocumentation

\medskip

\hrule

\medskip

\newpage

\phantomsection
\hypertarget{matoc}{}

\tableofcontents

\vfill

\section{Historique}

\verb|v0.1.0|~:~~~~Version initiale.

\newpage

\section{Le package postit}

\subsection{Introduction}

\begin{noteblock}
Le package propose de quoi afficher, dans son document \LaTeX, un Post-It (créé à l'aide de \texttt{tcolorbox}), avec la possibilité :

\begin{itemize}
	\item de spécifier les dimensions, la couleur ;
	\item de rajouter une \textit{attache} comme un trombone ou une punaise ;
	\item de personnaliser les bordure et le coin.
\end{itemize}

Le package propose également de quoi créer un \textit{mini-}Post-It (créé à l'aide d'une \texttt{tcbox}), avec la possibilité de gérer la couleur et l'ombre.
\end{noteblock}

\subsection{Chargement du package, packages utilisés}

\begin{importantblock}
Le package se charge, de manière classique, dans le préambule.

Il n'existe pas d'option pour le package, et \texttt{xcolor} n'est pas chargé.
\end{importantblock}

\begin{DemoCode}[listing only]
\documentclass{article}
\usepackage{postit}

\end{DemoCode}

\begin{noteblock}
\textsf{postit} charge les packages suivantes :

\begin{itemize}
	\item \texttt{tcolorbox} avec la librairie \texttt{\textit{tcbox}.skins} ;
	\item \texttt{xstring} et \texttt{simplekv}.
\end{itemize}

Il est compatible avec les compilations usuelles en \textsf{latex}, \textsf{pdflatex}, \textsf{lualatex} ou \textsf{xelatex}.
\end{noteblock}

\subsection{Compatibilité}

\begin{cautionblock}
Si un autre package charge \texttt{tcolorbox}, et notamment avec l'option \Cle{[most]}, il vaut mieux charger \texttt{postit} après, afin d'éviter un \textsf{option clash error\ldots}.
\end{cautionblock}

\begin{DemoCode}[listing only]
\documentclass{article}
\usepackage[<librairies>]{tcolorbox}
\usepackage{postit}
...

\end{DemoCode}

\vfill~

\pagebreak

\section{Environnement Post-It}

\subsection{Environnement et fonctionnement global}

\begin{cautionblock}
L'environnement dédié à la création du Post-It est \texttt{PostIt}.

Il fonctionne avec un système de clés, entre \texttt{[...]} et il est possible, entre \texttt{<...>} de spécifier des options à la \textsf{tcbox}, en langage \textsf{tcbox} !
\end{cautionblock}

\begin{DemoCode}[listing only]
\begin{PostIt}[clés]<options tcbox>
...
...
\end{PostIt}
\end{DemoCode}

\begin{noteblock}
Comme indiqué dans l'introduction, le Post-It est créé à l'aide d'un environnement \textsf{tcbox}.

La majorité des (multiples) paramètres d'une \textsf{tcbox} sont fixés par le code, mais il est possible de spécifier certaines caractéristiques esthétiques du Post-It !
\end{noteblock}

\begin{DemoCode}[]
%sortie par défaut, avec un paragraphe issu du package lipsum
\begin{PostIt}
\lipsum[1][1-2]
\end{PostIt}
\end{DemoCode}

\begin{tipblock}
Les éventuelles couleurs choisies devront être données de manière \textit{unique}, sans utiliser les \textit{mélanges} (avec \texttt{CouleurA!...!CouleurB}) que propose le package \texttt{xcolor}. 

Toutefois, toute couleur précédemment définie pourra être utilisée pour le Post-It.
\end{tipblock}

\begin{tipblock}
Le Post-It créé pourra être intégré dans une \textsf{minipage} ou un \textsf{wrapstuff} si besoin.

Pour l'alignement horizontal, il est conseillé d'utiliser des commandes dédiées comme \texttt{\textbackslash hfill} ou des envrionnements dédiées comme \texttt{flush...}.
\end{tipblock}

\subsection{Clés et options}

\begin{tipblock}
Le premier argument, optionnel et entre \texttt{[...]}, propose les \Cle{clés} suivantes :

\begin{itemize}
	\item \Cle{Largeur} : largeur du Post-It  ; \hfill{}défaut : \Cle{6cm}
	\item \Cle{Couleur} : couleur du Post-It (la bordure sera plus foncée) ; \hfill{}défaut : \Cle{yellow}
	\item \Cle{Hauteur} : hauteur du Post-It (par défaut elle est \textit{automatique}) ; \hfill{}défaut : \Cle{auto}
	\item \Cle{Inclinaison} : inclinaison du Post-It ; \hfill{}défaut : \Cle{0}
	\item \Cle{Ombre} : booléen pour afficher une ombre portée ; \hfill{}défaut : \Cle{true}
	\item \Cle{Bordure} : booléen pour afficher une fine bordure ; \hfill{}défaut : \Cle{true}
	\item \Cle{Coin} : booléen pour afficher un coin corné ; \hfill{}défaut : \Cle{false}
	\item \Cle{Attache} : choix de la décoration, parmi \Cle{Trombone / Punaise / Non} ;
	
	\hfill{}défaut : \Cle{Punaise}
	\item \Cle{CouleurAttache} : couleur de l'attache ; \hfill{}défaut : \Cle{red}
	\item \Cle{DecalAttache} : décalage horizontal de l'attache par rapport à sa position initiale (au centre pour la punaise, à 1~cm du bord droit pour le trombone) ;
	
	\hfill{}défaut : \Cle{0cm}
	\item \Cle{AlignementV} : gère l'alignement vertical dans le Post-It (parmi \Cle{top/center/bottom}) ;
	
	\hfill{}défaut : \Cle{top}
	\item \Cle{AlignementH} : gère l'alignement horizontal dans le Post-It (parmi \Cle{left/center/right/justify}) ;
	
	\hfill{}défaut : \Cle{justify}
	\item \Cle{AlignementPostIt} : gère l'alignement vertical global du Post-It (parmi \Cle{top/center/bottom}).
	
	\hfill{}défaut : \Cle{bottom}
\end{itemize}
\vspace*{-\baselineskip}\leavevmode
\end{tipblock}

\begin{tipblock}
Le second argument, optionnel et entre \texttt{<...>} correspond à des options spécifiques à passer à la \textsf{tcolorbox}, en langage \textsf{tcbox}.

Elles permettent de modifier localement des options non paramétrées par des clés présentées précédemment.
\end{tipblock}

\begin{DemoCode}[]
\begin{PostIt}
	[Couleur=cyan,Attache=Trombone,Largeur=10cm,Inclinaison=10]<center,right=1.5cm>
\lipsum[1][1-3]
\end{PostIt}
\end{DemoCode}

\pagebreak

\subsection{Exemples}

\begin{DemoCode}[]
%usepackage{wrapstuff}
\begin{wrapstuff}[r,top=1]
\begin{PostIt}[Inclinaison=5,Coin,Couleur=pink,CouleurAttache=blue,Bordure=false]
\lipsum[1][1-2]
\end{PostIt}
\end{wrapstuff}

\lipsum[1]
\end{DemoCode}

\begin{DemoCode}[]
Un petit Post-It aligné à droite, et centré verticalement :
%
\hfill\begin{PostIt}[Inclinaison=-10,Couleur=orange,Largeur=5cm,Hauteur=5cm, AlignementV=center,Coin,CouleurAttache=yellow, DecalAttache=-1cm,AlignementPostIt=center]

\textsf{\small\lipsum[1][1-2]}
\[\mathsf{\displaystyle\sum_{k=1}^{n} k = \dfrac{n(n+1)}{2}}\]

\end{PostIt}
\end{DemoCode}

\vfill~

\pagebreak

\section{Post-It simple, en ligne}

\subsection{Commande et fonctionnement global}

\begin{cautionblock}
La commande dédiée à la création du \textit{mini-}Post-It est \texttt{MiniPostIt}.

Elle fonctionne sous forme autonome, avec uniquement la couleur en \Cle{option}.

\smallskip

Cette fois-ci le \textit{mini-} Post-It est créé à l'aide d'une commande \textsf{tcbox}.
\end{cautionblock}

\begin{DemoCode}[listing only]
\MiniPostIt(*)[couleur]{contenu}
\end{DemoCode}

\subsection{Arguments}

\begin{noteblock}
La version étoilée active l'ombre du \textit{mini-}Post-It.

La couleur (\Cle{yellow}), est gérée par l'argument optionnel entre \texttt{[...]}.
\end{noteblock}

\subsection{Exemples}

\begin{DemoCode}[]
On va travailler sur une équation diophantienne du type $ax+by=c$.

On va utiliser le \MiniPostIt*[orange]{théorème de Bezout} puis le \MiniPostIt{théorème de Gauss}.

Le schéma de résolution est classique, et assez simple à appréhender !
\end{DemoCode}

\end{document}